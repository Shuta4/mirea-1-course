\documentclass[14pt, a4paper]{extreport}
\usepackage{extsizes}
\usepackage[a4paper, left=30mm, right=15mm, top=20mm, bottom=20mm]{geometry}
\usepackage[english, russian]{babel}
\usepackage{fontspec}
\defaultfontfeatures{Ligatures={TeX},Renderer=Basic}
\setmainfont[Ligatures={TeX,Historic}]{Times New Roman}
\usepackage{amsmath,amssymb}
\usepackage{graphicx}
\usepackage{setspace}
\graphicspath{{images/}}
\usepackage[hidelinks]{hyperref}
\usepackage{indentfirst}
\setlength{\parindent}{1.25cm}
\usepackage[explicit,compact]{titlesec}
\usepackage{titletoc}
\newcommand{\doublerule}[1][.4pt]{%
	\noindent
	\makebox[0pt][l]{\rule[.6ex]{\linewidth}{#1}}%
	\rule[.3ex]{\linewidth}{#1}
}

\addto\captionsrussian{%
	\renewcommand{\contentsname}%
	{\centering{СОДЕРЖАНИЕ}}%
}

\usepackage{caption}
\DeclareCaptionLabelSeparator{dash}{ -- }
\DeclareCaptionLabelFormat{figure}{Рисунок #2}
\captionsetup[table]{
	labelsep=dash,
	singlelinecheck=false,
}
\captionsetup[figure]{
	labelsep=dash,
	labelformat=figure,
}

\usepackage{floatrow}
\floatsetup[table]{style=plaintop}
\floatsetup[equation]{style=plain}

\usepackage{chngcntr}
\counterwithout{figure}{chapter}
\counterwithout{equation}{chapter}
\counterwithout{table}{chapter}

\usepackage{cleveref}
\crefformat{table}{смотри табл.#2#1#3}
\Crefformat{table}{Смотри табл.#2#1#3}

\crefformat{figure}{рис.~#2#1#3}
\Crefformat{figure}{Рис.~#2#1#3}
\crefmultiformat{figure}{рис.~#2#1#3}{,~#2#1#3}{,~#2#1#3}{,~#2#1#3}
\crefrangeformat{figure}{рис.~#3#1#4--#5#2#6}

\crefformat{equation}{#1}
\crefmultiformat{equation}{~#2#1#3}{,~#2#1#3}{,~#2#1#3}{,~#2#1#3}
\crefrangeformat{equation}{~#3#1#4--#5#2#6}

\begin{document}
\begin{titlepage}
	\begin{center}
		\vspace*{0.5mm}
		\setstretch{1.1}

		\includegraphics[width=0.18\textwidth]{logo}\\
		\footnotesize
		МИНИСТЕРСТВО НАУКИ И ВЫСШЕГО ОБРАЗОВАНИЯ РОССИЙСКОЙ ФЕДЕРАЦИИ\\
		\small
		Федеральное государственное бюджетное образовательное учреждение высшего образования\\
		\textbf{«МИРЭА - Российский технологический университет»}
		\vspace{0.5cm}

		\large \textbf{РТУ МИРЭА} \normalsize

		\doublerule[1pt]\\
		\vspace{0.4cm}

		Институт искусственного интеллекта\\
		Кафедра общей информатики
		\vspace{1.5cm}

		\textbf{ОТЧЕТ}\\
		\textbf{ПО ПРАКТИЧЕСКОЙ РАБОТЕ № 11}\\
		\textbf{синтез четырехразрядного счетчика с параллельным переносом между разрядами двумя способами}\\
		\textbf{по дисциплине}\\
		«ИНФОРМАТИКА»
		\vspace{1.5cm}

		\small
		Выполнил студент группы ИМБО-01-22 \hfill Скирдин Никита Сергеевич
		\vspace{1cm}

		Принял \hfill Павлова Екатерина Сергеевна\\
		ассистент \hfill
		\vspace{1.5cm}

		\footnotesize
		\hspace{0.5cm} Практическая \hfill «\_\_»\_\_\_\_\_\_2022 г. \hfill Подпись студента\\
		\hspace{0.5cm} работа выполнена \hfill
		\vspace{0.5cm}

		\hspace{2cm} «Зачтено» \hfill «\_\_»\_\_\_\_\_\_2022 г. \hfill Подпись преподавателя
		\vfill

		\small
		Москва 2022
	\end{center}
	\thispagestyle{empty}
\end{titlepage}

\setstretch{1.5}
\setlength{\abovedisplayskip}{0.1em}
\setlength{\belowdisplayskip}{0.1em}
\setlength{\abovedisplayshortskip}{0pt}
\setlength{\belowdisplayshortskip}{0pt}
\setlength{\floatsep}{1em}
\setlength{\textfloatsep}{1em}
\setlength{\intextsep}{1em}
\setcounter{page}{2}

\titlecontents{chapter}[0em]
	{\vskip 0.5ex}%
	{\thecontentslabel \space \uppercase}% numbered sections formatting
	{}% unnumbered sections formatting
	{\hfill \thecontentspage}%

\titlecontents{section}[1.25cm]
	{\vskip 0.5ex}%
	{\thecontentslabel \space}
	{}
	{\hfill \thecontentspage}

\titleformat{\chapter}[block]
	{\bfseries\normalsize}{}{0pt}{\uppercase{#1}}

\titleformat{\section}[block]
	{\bfseries\normalsize}{}{0pt}{#1}

\titlespacing*{\chapter}{0pt}{-10.5mm}{0pt}

\tableofcontents

\titleformat{\chapter}[display]
	{\centering\bfseries\normalsize}{}{0pt}{\thechapter \space \uppercase{#1}}

\titleformat{\section}[block]
	{\hspace{\parindent}\bfseries\normalsize}{}{0pt}{\thesection \space #1}

\titlespacing*{\chapter}{0pt}{-19.5mm}{0pt}

\chapter{Постановка задачи}
Разработать счетчик с параллельным переносом на D-триггерах двумя способами:

-- с оптимальной схемой управления, выполненной на логических элементах общего базиса;

-- со схемой управления, реализованной на преобразователе кодов (быстрая реализация, но не оптимальная схема).

Протестировать работу схемы и убедиться в ее правильности. Подготовить отчет о проделанной работе и защитить ее.

\section{Персональный вариант}
Исходные данные:

-- направление счета: вычитание;

-- максимальное значение счетчика: 9;

-- шаг счета: 6;

\chapter{Проектирование и реализация}
\section{Предварительная подготовка данных}
По исходным данным восстановим таблицу переходов счетчика (\cref{tab:states}).

\begin{table}[H]
	\caption{Таблица переходов счетчика}
	\label{tab:states}
	\begin{tabular}{|c|c|c|c|c|c|c|c|}
		\hline
		$Q_3(t)$ & $Q_2(t)$ & $Q_1(t)$ & $Q_0(t)$ & $Q_3(t + 1)$ & $Q_2(t + 1)$ & $Q_1(t + 1)$ & $Q_0(t + 1)$ \\
		\hline
		0 & 0 & 0 & 0 & 0 & 1 & 0 & 0 \\
		\hline
		0 & 0 & 0 & 1 & 0 & 0 & 0 & 1 \\
		\hline
		0 & 0 & 1 & 0 & 0 & 1 & 1 & 0 \\
		\hline
		0 & 0 & 1 & 1 & 0 & 1 & 1 & 1 \\
		\hline
		0 & 1 & 0 & 0 & 1 & 0 & 0 & 0 \\
		\hline
		0 & 1 & 0 & 1 & 1 & 0 & 0 & 1 \\
		\hline
		0 & 1 & 1 & 0 & 0 & 0 & 0 & 0 \\
		\hline
		0 & 1 & 1 & 1 & 0 & 0 & 0 & 1 \\
		\hline
		1 & 0 & 0 & 0 & 0 & 0 & 1 & 0 \\
		\hline
		1 & 0 & 0 & 1 & 0 & 0 & 1 & 1 \\
		\hline
		1 & 0 & 1 & 0 & * & * & * & * \\
		\hline
		1 & 0 & 1 & 1 & * & * & * & * \\
		\hline
		1 & 1 & 0 & 0 & * & * & * & * \\
		\hline
		1 & 1 & 0 & 1 & * & * & * & * \\
		\hline
		1 & 1 & 1 & 0 & * & * & * & * \\
		\hline
		1 & 1 & 1 & 1 & * & * & * & * \\
		\hline
	\end{tabular}
\end{table}

Таблица переходов является частично определенной: состояния с 1010 по 1111, согласно исходным данным, возникать никогда не должны.

\section{Минимизация функций при помощи карт Карно}
Рассматриваем столбцы $Q_i(t + 1)$ как самостоятельные функции от четырех переменных и проводим их минимизацию. Также необходимо для каждой функции из двух возможных минимальных форм выбрать самую короткую.

Начнем с функции $Q_3(t + 1)$. Оценим сложность минимальных форм, которые для нее получатся, по количеству переменных, входящих в них, и выберем оптимальную форму. Для этого построим необходимые карты Карно.

На \cref{fig:map-mdnf-q3} показана карта для МДНФ функции $Q_3(t + 1)$.

\begin{figure}[H]
	\caption{Карта Карно для МДНФ функции $Q_3(t + 1)$}
	\label{fig:map-mdnf-q3}
	\includegraphics[width=0.5\textwidth]{map-mdnf-q3-selected}
\end{figure}

Из \cref{fig:map-mdnf-q3} видно, что МДНФ $Q_3(t + 1)$ будет описана при помощи 2 переменных либо их отрицаний.

Теперь проделаем аналогичную операцию для МКНФ этой же функции. Из \cref{fig:map-mknf-q3} видно, что МКНФ $Q_3(t + 1)$ будет описана при помощи $1 + 2 = 3$ переменных либо их отрицаний.

\begin{figure}[H]
	\caption{Карта Карно для МКНФ функции $Q_3(t + 1)$}
	\label{fig:map-mknf-q3}
	\includegraphics[width=0.5\textwidth]{map-mknf-q3-selected}
\end{figure}

Запишем МДНФ для $Q_3(t + 1)$ (\cref{equ:q3}).
\begin{align}
	\label{equ:q3}
	Q_3(t + 1)_\textup{МДНФ} = Q_2(t) \cdot \overline{Q_1(t)}
\end{align}

По такой же методике рассуждений рассмотрим функцию $Q_2(t + 1)$. Из \cref{fig:map-mdnf-q2} видно, что МДНФ $Q_2(t + 1)$ будет описана при помощи $3 + 2 = 5$ переменных либо их отрицаний.

\begin{figure}[H]
	\caption{Карта Карно для МДНФ функции $Q_2(t + 1)$}
	\label{fig:map-mdnf-q2}
	\includegraphics[width=0.5\textwidth]{map-mdnf-q2-selected}
\end{figure}

Теперь проделаем аналогичную операцию для МКНФ этой же функции. Из \cref{fig:map-mknf-q2} видно, что МКНФ $Q_2(t + 1)$ будет описана при помощи $1 + 1 + 2 = 4$ переменных либо их отрицаний.

\begin{figure}[H]
	\caption{Карта Карно для МКНФ функции $Q_2(t + 1)$}
	\label{fig:map-mknf-q2}
	\includegraphics[width=0.5\textwidth]{map-mknf-q2-selected}
\end{figure}

Запишем МКНФ для $Q_2(t + 1)$ (\cref{equ:q2}).
\begin{align}
	\label{equ:q2}
	Q_2(t + 1)_\textup{МКНФ} = \overline{Q_2(t)} \cdot \overline{Q_3(t)} \cdot (Q_1(t) + \overline{Q_0(t)})
\end{align}

Переходим к рассмотрению $Q_1(t + 1)$. Из \cref{fig:map-mdnf-q1} видно, что МДНФ $Q_1(t + 1)$ будет описана при помощи $1 + 2 = 3$ переменных либо их отрицаний.

\begin{figure}[H]
	\caption{Карта Карно для МДНФ функции $Q_1(t + 1)$}
	\label{fig:map-mdnf-q1}
	\includegraphics[width=0.5\textwidth]{map-mdnf-q1-selected}
\end{figure}

Теперь проделаем аналогичную операцию для МКНФ этой же функции. Из \cref{fig:map-mknf-q1} видно, что МКНФ $Q_1(t + 1)$ будет описана при помощи $1 + 2 = 3$ переменных либо их отрицаний.

\begin{figure}[H]
	\caption{Карта Карно для МКНФ функции $Q_1(t + 1)$}
	\label{fig:map-mknf-q1}
	\includegraphics[width=0.5\textwidth]{map-mknf-q1-selected}
\end{figure}

Запишем МДНФ для $Q_1(t + 1)$ (\cref{equ:q1}).
\begin{align}
	\label{equ:q1}
	Q_1(t + 1)_\textup{МДНФ} = Q_3(t) + \overline{Q_2(t)} \cdot Q_1(t)
\end{align}

\section{Реализация счетчика с оптимальной схемой управления}
Для $Q_0(t + 1)$ не требуется составление схемы управления, потому что, как видно из таблицы переходов (\cref{tab:states}), $Q_0(t + 1) = Q_0(t)$, поэтому просто подключаем выход триггера $Q_0$ к его входу D.

При помощи полученных формул выполним реализацию схем управления для триггеров счетчика (\cref{fig:mdnf-mknf}). Тестирование показало, что схемы работают правильно.

\begin{figure}[H]
	\caption{Схема счетчика на основе МДНФ и МКНФ}
	\label{fig:mdnf-mknf}
	\includegraphics[width=\textwidth]{mdnf-mknf}
\end{figure}

\section{Реализация счетчика на преобразователе кодов}
Для реализации счетчика при помощи преобразователя кодов в качестве схемы управления триггерами не требуется никакая минимизация, необходимо просто по таблице переходов правильно соединить выходы дешифратора со входами шифратора. Таким образом, можно сразу построить схему счетчика (\cref{fig:coder}). Тестирование показало, что схема работает правильно.

\begin{figure}[H]
	\caption{Счетчик со схемой управления на преобразователе кодов}
	\label{fig:coder}
	\includegraphics[width=\textwidth]{coder}
\end{figure}

\chapter{Выводы}
В ходе работы был разработан счетчик с параллельным переносом на D-триггерах двумя способами. Работа схем была протестирована, чтобы убедиться в их правильности.

\chapter{Информационный источник}
\textbf{Смирнов, С. С.} Информатика : Методические указания по выполнению практических работ / С. С. Смирнов, Д. А. Карпов. -- Москва : МИРЭА -- Российский технологический университет, 2020. -- 102 с.

\end{document}

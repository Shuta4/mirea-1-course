\documentclass[14pt, a4paper]{extreport}
\usepackage{extsizes}
\usepackage[a4paper, left=30mm, right=15mm, top=20mm, bottom=20mm]{geometry}
\usepackage[english, russian]{babel}
\usepackage{fontspec}
\defaultfontfeatures{Ligatures={TeX},Renderer=Basic}
\setmainfont[Ligatures={TeX,Historic}]{Times New Roman}
\usepackage{amsmath,amssymb}
\usepackage{graphicx}
\usepackage{setspace}
\graphicspath{{images/}}
\usepackage[hidelinks]{hyperref}
\usepackage{indentfirst}
\setlength{\parindent}{1.25cm}
\usepackage[explicit,compact]{titlesec}
\usepackage{titletoc}
\newcommand{\doublerule}[1][.4pt]{%
	\noindent
	\makebox[0pt][l]{\rule[.6ex]{\linewidth}{#1}}%
	\rule[.3ex]{\linewidth}{#1}
}

\addto\captionsrussian{%
	\renewcommand{\contentsname}%
	{\centering{СОДЕРЖАНИЕ}}%
}

\usepackage{caption}
\DeclareCaptionLabelSeparator{dash}{ -- }
\DeclareCaptionLabelFormat{figure}{Рисунок #2}
\captionsetup[table]{
	labelsep=dash,
	singlelinecheck=false,
}
\captionsetup[figure]{
	labelsep=dash,
	labelformat=figure,
}

\usepackage{floatrow}
\floatsetup[table]{style=plaintop}
\floatsetup[equation]{style=plain}

\usepackage{chngcntr}
\counterwithout{figure}{chapter}
\counterwithout{equation}{chapter}
\counterwithout{table}{chapter}

\usepackage{cleveref}
\crefformat{table}{смотри табл.#2#1#3}
\Crefformat{table}{Смотри табл.#2#1#3}

\crefformat{figure}{рис.~#2#1#3}
\Crefformat{figure}{Рис.~#2#1#3}
\crefmultiformat{figure}{рис.~#2#1#3}{,~#2#1#3}{,~#2#1#3}{,~#2#1#3}
\crefrangeformat{figure}{рис.~#3#1#4--#5#2#6}

\crefformat{equation}{#1}
\crefmultiformat{equation}{~#2#1#3}{,~#2#1#3}{,~#2#1#3}{,~#2#1#3}
\crefrangeformat{equation}{~#3#1#4--#5#2#6}

\begin{document}
\begin{titlepage}
	\begin{center}
		\vspace*{0.5mm}
		\setstretch{1.1}

		\includegraphics[width=0.18\textwidth]{logo}\\
		\footnotesize
		МИНИСТЕРСТВО НАУКИ И ВЫСШЕГО ОБРАЗОВАНИЯ РОССИЙСКОЙ ФЕДЕРАЦИИ\\
		\small
		Федеральное государственное бюджетное образовательное учреждение высшего образования\\
		\textbf{«МИРЭА - Российский технологический университет»}
		\vspace{0.5cm}

		\large \textbf{РТУ МИРЭА} \normalsize

		\doublerule[1pt]\\
		\vspace{0.4cm}

		Институт искусственного интеллекта\\
		Кафедра общей информатики
		\vspace{1.5cm}

		\textbf{ОТЧЕТ}\\
		\textbf{ПО ПРАКТИЧЕСКОЙ РАБОТЕ № 6}\\
		\textbf{построение комбинационных схем, реализующих МДНФ и МКНФ заданной логической функции от 4-х переменных в базисах И-НЕ, ИЛИ-НЕ}\\
		\textbf{по дисциплине}\\
		«ИНФОРМАТИКА»
		\vspace{1.5cm}

		\small
		Выполнил студент группы ИМБО-01-22 \hfill Скирдин Никита Сергеевич
		\vspace{1cm}

		Принял \hfill Павлова Екатерина Сергеевна\\
		ассистент \hfill
		\vspace{1.5cm}

		\footnotesize
		\hspace{0.5cm} Практическая \hfill «\_\_»\_\_\_\_\_\_2022 г. \hfill Подпись студента\\
		\hspace{0.5cm} работа выполнена \hfill
		\vspace{0.5cm}

		\hspace{2cm} «Зачтено» \hfill «\_\_»\_\_\_\_\_\_2022 г. \hfill Подпись преподавателя
		\vfill

		\small
		Москва 2022
	\end{center}
	\thispagestyle{empty}
\end{titlepage}

\setstretch{1.5}
\setlength{\abovedisplayskip}{0.1em}
\setlength{\belowdisplayskip}{0.1em}
\setlength{\abovedisplayshortskip}{0pt}
\setlength{\belowdisplayshortskip}{0pt}
\setlength{\floatsep}{1em}
\setlength{\textfloatsep}{1em}
\setlength{\intextsep}{1em}
\setcounter{page}{2}

\titlecontents{chapter}[0em]
	{\vskip 0.5ex}%
	{\thecontentslabel \space \uppercase}% numbered sections formatting
	{}% unnumbered sections formatting
	{\hfill \thecontentspage}%

\titlecontents{section}[1.25cm]
	{\vskip 0.5ex}%
	{\thecontentslabel \space}
	{}
	{\hfill \thecontentspage}

\titleformat{\chapter}[block]
	{\bfseries\normalsize}{}{0pt}{\uppercase{#1}}

\titleformat{\section}[block]
	{\bfseries\normalsize}{}{0pt}{#1}

\titlespacing*{\chapter}{0pt}{-10.5mm}{0pt}

\tableofcontents

\titleformat{\chapter}[display]
	{\centering\bfseries\normalsize}{}{0pt}{\thechapter \space \uppercase{#1}}

\titleformat{\section}[block]
	{\hspace{\parindent}\bfseries\normalsize}{}{0pt}{\thesection \space #1}

\titlespacing*{\chapter}{0pt}{-19.5mm}{0pt}

\chapter{Постановка задачи}
Логическая функция от четырех переменных задана в 16-теричной векторной форме. Восстановить таблицу истинности. Минимизировать логическую функцию при помощи карт Карно и получить формулы МДНФ и МКНФ в общем базисе. Перевести МДНФ и МКНФ в базисы «И-НЕ» и «ИЛИ-НЕ» (каждую минимальную форму в два базиса). Построить комбинационные схемы для приведенных к базисам формул МДНФ и МКНФ в лабораторном комплексе, используя только логические элементы, входящие в конкретный базис. Протестировать работу схем и убедиться в их правильности. Подготовить отчет о проделанной работе и защитить ее.

\section{Персональный вариант}
Логическая функция от четырех переменных, заданная в 16-теричной форме: F9AA$_{16}$

\chapter{Проектирование и реализация}
\section{Предварительная подготовка данных}
Преобразуем заданную логическую функцию в двоичную запись: 1111 1001 1010 1010$_2$ - получили столбец значений логической функции, который необходим для восстановления полной таблицы истинности (\cref{tab:function-values}).

\begin{table}[!htbp]
	\caption{Таблица истинности заданной функции}
	\label{tab:function-values}
	\begin{tabular}{|c|c|c|c|c|}
		\hline
		a & b & c & d & F \\
		\hline
		0 & 0 & 0 & 0 & 1 \\
		\hline
		0 & 0 & 0 & 1 & 1 \\
		\hline
		0 & 0 & 1 & 0 & 1 \\
		\hline
		0 & 0 & 1 & 1 & 1 \\
		\hline
		0 & 1 & 0 & 0 & 1 \\
		\hline
		0 & 1 & 0 & 1 & 0 \\
		\hline
		0 & 1 & 1 & 0 & 0 \\
		\hline
		0 & 1 & 1 & 1 & 1 \\
		\hline
		1 & 0 & 0 & 0 & 1 \\
		\hline
		1 & 0 & 0 & 1 & 0 \\
		\hline
		1 & 0 & 1 & 0 & 1 \\
		\hline
		1 & 0 & 1 & 1 & 0 \\
		\hline
		1 & 1 & 0 & 0 & 1 \\
		\hline
		1 & 1 & 0 & 1 & 0 \\
		\hline
		1 & 1 & 1 & 0 & 1 \\
		\hline
		1 & 1 & 1 & 1 & 0 \\
		\hline
	\end{tabular}
\end{table}

\section{Вывод формулы для МДНФ}
Для построения МДНФ заданной функции воспользуемся методом карт Карно. Разместим единичные значения функции на карте Карно, предназначенной для минимизации функции от четырех переменных (\cref{fig:map-mdnf}).

\begin{figure}[H]
	\caption{Карта Карно, заполненная для построения МДНФ}
	\label{fig:map-mdnf}
	\includegraphics[width=0.5\textwidth]{map-mdnf}\\
\end{figure}

Теперь выделяем интервалы, на которых функция сохраняет свое единичное значение, результат выделения интервалов показан на \cref{fig:map-mdnf-selected}.

\begin{figure}[H]
	\caption{Результат выделения интервалов для МДНФ}
	\label{fig:map-mdnf-selected}
	\includegraphics[width=0.5\textwidth]{map-mdnf-selected}
\end{figure}

Далее запишем формулу МДНФ, для чего последовательно рассмотрим каждый из интервалов. Для каждого интервала запишем минимальную конъюнкцию, куда будут входить только те переменные и их отрицания, которые сохраняют свое значение на этом интервале. Переменные, которые меняют свое значение на интервале, упростятся. Чтобы получить МДНФ остается только объединить при помощи дизъюнкции имеющееся множество минимальных конъюнкций. В результате получим формулу \cref{equ:mdnf}.
\begin{align}
	\label{equ:mdnf}
	F_\textup{МДНФ} = \bar{a} \cdot \bar{b} + \bar{c} \cdot \bar{d} + %
					  \bar{a} \cdot c \cdot d + a \cdot c \cdot \bar{d}
\end{align}

Теперь приведем полученную МДНФ к базисам «И-НЕ» и «ИЛИ-НЕ», для чего воспользуемся законами де Моргана. В результате получим формулы \cref{equ:mdnf-and-not,equ:mdnf-or-not}.
\begin{align}
	\label{equ:mdnf-and-not}
	F_{\textup{МДНФ}_\textup{и-не}} = \overline{%
		\overline{\overline{a} \cdot \overline{b}} \cdot %
		\overline{\overline{c} \cdot \overline{d}} \cdot %
		\overline{\overline{a} \cdot c \cdot d} \cdot %
		\overline{a \cdot c \cdot \overline{d}}%
	}
\end{align}
\begin{align}
	\label{equ:mdnf-or-not}
	F_{\textup{МДНФ}_\textup{или-не}} = \overline{\overline{%
		\overline{a + b} + \overline{c + d} + %
		\overline{a + \overline{c} + \overline{d}} + %
		\overline{\overline{a} + \overline{c} + d}
	}}
\end{align}

\section{Вывод формулы для МКНФ}

Для построения МКНФ заданной функции воспользуемся методом карт Карно. Разместим нулевые значения функции на карте Карно, предназначенной для минимизации функции от четырех переменных (\cref{fig:map-mknf}).

\begin{figure}[H]
	\caption{Карта Карно, заполненная для построения МКНФ}
	\label{fig:map-mknf}
	\includegraphics[width=0.5\textwidth]{map-mknf}
\end{figure}

Теперь выделяем интервалы, на которых функция сохраняет свое нулевое значение, результат выделения интервалов показан на \cref{fig:map-mknf-selected}.

\begin{figure}[H]
	\caption{Результат выделения интервалов для МКНФ}
	\label{fig:map-mknf-selected}
	\includegraphics[width=0.5\textwidth]{map-mknf-selected}
\end{figure}

Далее запишем формулу МКНФ, для чего последовательно рассмотрим каждый из интервалов. Для каждого интервала запишем минимальную дизъюнкцию, куда будут входить только те переменные и их отрицания, которые сохраняют свое значение на этом интервале. Переменные, которые меняют свое значение на интервале, упростятся. Чтобы получить МКНФ остается только объединить при помощи конъюнкции имеющееся множество минимальных дизъюнкций. В результате получим формулу \cref{equ:mknf}.
\begin{align}
	\label{equ:mknf}
	F_\textup{МКНФ} = (\bar{a} + \bar{d}) \cdot (\bar{b} + c + \bar{d})%
		\cdot (a + \bar{b} + \bar{c} + d)
\end{align}

Теперь приведем полученную МДНФ к базисам «И-НЕ» и «ИЛИ-НЕ», для чего воспользуемся законами де Моргана. В результате получим формулы \cref{equ:mknf-and-not,equ:mknf-or-not}.
\begin{align}
	\label{equ:mknf-and-not}
	F_{\textup{МКНФ}_\textup{и-не}} = \overline{\overline{%
		\overline{a \cdot d} \cdot%
		\overline{b \cdot \overline{c} \cdot d} \cdot%
		\overline{\overline{a} \cdot b \cdot c \cdot \overline{d}}%
	}}
\end{align}
\begin{align}
	\label{equ:mknf-or-not}
	F_{\textup{МКНФ}_\textup{или-не}} = \overline{%
		\overline{\overline{a} + \overline{d}} +%
		\overline{\overline{b} + c + \overline{d}} +%
		\overline{a + \overline{b} + \overline{c} + d}%
	}
\end{align}

\section{Построение схем в лабораторном комплексе}
Построим в лабораторном комплексе комбинационные схемы, реализующие МДНФ и МКНФ рассматриваемой функции в базисах «И-НЕ» и «ИЛИ-НЕ», протестируем их работу и убедимся в их правильности (\cref{fig:mdnf-and-not,fig:mdnf-or-not,fig:mknf-and-not,fig:mknf-or-not}).

\begin{figure}[H]
	\centering
	\caption{Тестирование схемы МДНФ, построенной в базисе «И-НЕ»}
	\label{fig:mdnf-and-not}
	\includegraphics[width=\textwidth]{mdnf-and-not}
\end{figure}

\begin{figure}[H]
	\centering
	\caption{Тестирование схемы МДНФ, построенной в базисе «ИЛИ-НЕ»}
	\label{fig:mdnf-or-not}
	\includegraphics[width=\textwidth]{mdnf-or-not}
\end{figure}

\begin{figure}[H]
	\centering
	\caption{Тестирование схемы МКНФ, построенной в базисе «И-НЕ»}
	\label{fig:mknf-and-not}
	\includegraphics[width=\textwidth]{mknf-and-not}
\end{figure}

\begin{figure}[H]
	\centering
	\caption{Тестирование схемы МКНФ, построенной в базисе «ИЛИ-НЕ»}
	\label{fig:mknf-or-not}
	\includegraphics[width=\textwidth]{mknf-or-not}
\end{figure}

\chapter{Выводы}
В ходе работы была восстановлена таблица истинности заданной логической функции от четырех переменных. Функция была минимизирована при помощи карт Карно, для нее были записаны формулы МДНФ и МКНФ в общем базисе. МДНФ и МКНФ были переведены в базисы «И-НЕ» и «ИЛИ-НЕ». В лабораторном комплексе были построены комбинационные схемы приведенных к базисам МДНФ и МКНФ с использованием элементов, входящих в конкретный базис. Работа схем была протестирована.

\chapter{Информационный источник}
\textbf{Смирнов, С. С.} Информатика : Методические указания по выполнению практических работ / С. С. Смирнов, Д. А. Карпов. -- Москва : МИРЭА -- Российский технологический университет, 2020. -- 102 с.

\end{document}
